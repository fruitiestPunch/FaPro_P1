% % -*- coding:utf-8 -*-
\documentclass[aspectratio=169,10pt]{beamer}
\nonstopmode

\usepackage{appendixnumberbeamer}
\usepackage{graphicx}
\usepackage{url}
\usepackage{amsmath, amssymb}
\usepackage{pifont} %fuer checkmarks
\input{colors}

% \usepackage{beamerthememetropolis}
\usetheme[progressbar=frametitle,noslidenumbers]{metropolis}
\newcommand{\themename}{\textbf{\textsc{metropolis}}\xspace}


\usepackage{xcolor}


\title{Abschlusspr\"asentation Projekt 1}
% \subtitle{Learning to Identify Similarities between Mathematical Expressions}
\author{Pouria Araghchi 170468, Kai Lukas Ilmenau 225338, Naveed Niazi 214471}

\institute{TU Dortmund - Fachprojekt zu "Routingalgorithmen"}
\titlegraphic{\hfill\includegraphics[height=8mm]{tu-do-logo.pdf}}

\date{\today}

\begin{document}

\maketitle
% %%%%%%%%%%%%%%%%%%%%%%%%%%%%%%%%%%%%%%%%%%%%%%%%%%%
% %%%%%%%%%%%%%%%%%%%%%%%%%%%%%%%%%%%%%%%%%%%%%%%%%%%
% Kai
% %%%%%%%%%%%%%%%%%%%%%%%%%%%%%%%%%%%%%%%%%%%%%%%%%%%
% %%%%%%%%%%%%%%%%%%%%%%%%%%%%%%%%%%%%%%%%%%%%%%%%%%%
\section{Inverse Capacity mit Zentralit\"atsmerkmalen}

\begin{frame}{Inverse Capacity mit Zentralit\"atsmerkmalen}
    \begin{itemize}
        \item inverseCapacity weights werden mit den Zentralit\"aten der Knoten einer Kante verrechnet
        \begin{itemize}
            \item $weight * \frac{CentralityNode_i + CentralityNode_j}{2}$
        \end{itemize}
        \item untersuchte Zentralit\"atensmetriken:
        \begin{itemize}
            \item Betweenness, Closeness, Eigenvevtor
        \end{itemize}
    \end{itemize}
\end{frame}

\begin{frame}{Welche Zentralit\"atsmetriken?}
\begin{columns}
\begin{column}[t]{0.3\paperwidth}
\textbf{Closeness} centrality
    \begin{itemize}
    \item wie nah ein Knoten zu den anderen ist
    \item $\mathcal{P}_{i\rightarrow j}$ ist der k\"urzeste Pfad von $i$ nach $j$
    \item $H(\mathcal{P}_{i\rightarrow j})$ ist der Hop-Count des Pfades
    \[
    c_i = \frac{1}{\sum_{j\neq i}H(\mathcal{P}_{i\rightarrow j})}
    \]
    \end{itemize}
\end{column}

\begin{column}[t]{0.3\paperwidth}
\textbf{Betweenness} centrality
    \begin{itemize}
    \item Verh\"altnis aller k\"urzeren Wege zur Anzahl der k\"urzesten Wege die durch den Knoten gehen
    \[
    b_i = \sum_{s,t\in\mathcal{N}}\frac{\lvert\mathcal{P}_{i\rightarrow j}(i)\rvert}{\lvert\mathcal{P}_{i\rightarrow j}\rvert} 
    \]
    \end{itemize}
\end{column}

\begin{column}[t]{0.3\paperwidth}
\textbf{Eigenvector} centrality
    \begin{itemize}
    \item entspricht dem $i$-ten Element des Eigenvektors der dem gr\"o\ss ten Eigenwert $\lambda_1$ der Adjazenzmatrix entspricht
    \end{itemize}
\end{column}
\end{columns}
\end{frame}

\begin{frame}[fragile]{Ergebnisse}
\includegraphics[width=.4\textwidth, angle=270]{images/kai_6.png}
\end{frame}
\begin{frame}[fragile]{Ergebnisse}
\includegraphics[width=.4\textwidth, angle=270]{images/kai_7.png}
\end{frame}
\begin{frame}[fragile]{Ergebnisse}
\includegraphics[width=.4\textwidth, angle=270]{images/kai_8.png}
\end{frame}

\begin{frame}[fragile]{Erkl\"arung}
\includegraphics[width=.7\textwidth]{images/kai_1.jpg}
\end{frame}
\begin{frame}[fragile]{InverseCapacity anwenden}
\includegraphics[width=.7\textwidth]{images/kai_2.jpg}
\end{frame}
\begin{frame}[fragile]{Optimale Pfade}
\includegraphics[width=.7\textwidth]{images/kai_3.jpg}
\end{frame}
\begin{frame}[fragile]{Jetzt mit Zentralit\"aten}
\includegraphics[width=.7\textwidth]{images/kai_4.jpg}
\end{frame}
\begin{frame}[fragile]{Anzahl der Pfade}
\includegraphics[width=.7\textwidth]{images/kai_5.jpg}
\end{frame}



% %%%%%%%%%%%%%%%%%%%%%%%%%%%%%%%%%%%%%%%%%%%%%%%%%%%
% %%%%%%%%%%%%%%%%%%%%%%%%%%%%%%%%%%%%%%%%%%%%%%%%%%%
% Naveed
% %%%%%%%%%%%%%%%%%%%%%%%%%%%%%%%%%%%%%%%%%%%%%%%%%%%
% %%%%%%%%%%%%%%%%%%%%%%%%%%%%%%%%%%%%%%%%%%%%%%%%%%%
\section{Naveed}
\begin{frame}[fragile]{Plotergebnisse}
\begin{itemize}
    \item \textbf{All-algorithms}
\end{itemize}
\begin{center}
    \includegraphics[width=.6\textwidth]{images/naveed_1.pdf}
\end{center}
\end{frame}
\begin{frame}[fragile]{Plotergebnisse}
\begin{itemize}
    \item \textbf{All-topologies:}\newline $1)$
\end{itemize}
\begin{center}
    \includegraphics[width=.7\textwidth]{images/naveed_3.pdf}
\end{center}
\end{frame}
\begin{frame}[fragile]{Plotergebnisse}
\begin{itemize}
    \item \textbf{All-topologies:}\newline $2)$
\end{itemize}
\begin{center}
    \includegraphics[width=.7\textwidth]{images/naveed_4.pdf}
\end{center}
\end{frame}
\begin{frame}[fragile]{Plotergebnisse}
\begin{itemize}
    \item \textbf{All-topologies:}\newline $3)$
\end{itemize}
\begin{center}
    \includegraphics[width=.5\textwidth]{images/naveed_2.pdf}
\end{center}
\end{frame}
\begin{frame}[fragile]{Plotergebnisse}
\begin{itemize}
    \item \textbf{All-topologies:}\newline $4)$
\end{itemize}
\begin{center}
    \includegraphics[width=.7\textwidth]{images/naveed_5.pdf}
\end{center}
\end{frame}
\begin{frame}[fragile]{Plotergebnisse}
\begin{itemize}
    \item \textbf{Real demands:}
\end{itemize}
\begin{center}
    \includegraphics[width=.6\textwidth]{images/naveed_6.pdf}
\end{center}
\end{frame}
\begin{frame}{Ergebnisse}
\begin{center}
    \includegraphics[width=\textwidth]{images/naveed_11.pdf}
\end{center}
\end{frame}
\begin{frame}{Ergebnisse}
\begin{center}
    \includegraphics[width=\textwidth]{images/naveed_12.pdf}
\end{center}
\end{frame}
\begin{frame}{Ergebnisse}
\begin{center}
    \includegraphics[width=\textwidth]{images/naveed_13.pdf}
\end{center}
\end{frame}
\begin{frame}{Ergebnisse}
\begin{center}
    \includegraphics[width=\textwidth]{images/naveed_14.pdf}
\end{center}
\end{frame}
\begin{frame}[fragile]{}
\begin{center}
    \includegraphics[width=\textwidth]{images/naveed_7.jpg}
\end{center}
\end{frame}

% %%%%%%%%%%%%%%%%%%%%%%%%%%%%%%%%%%%%%%%%%%%%%%%%
% Pouria
% %%%%%%%%%%%%%%%%%%%%%%%%%%%%%%%%%%%%%%%%%%%%%%%%

% %%%%%%%%%%%%%%%%%%%%%%%%%%%%%%%%%%%%%%%%%%%%%%%%%%%
% %%%%%%%%%%%%%%%%%%%%%%%%%%%%%%%%%%%%%%%%%%%%%%%%%%%
% Pouria
% %%%%%%%%%%%%%%%%%%%%%%%%%%%%%%%%%%%%%%%%%%%%%%%%%%%
% %%%%%%%%%%%%%%%%%%%%%%%%%%%%%%%%%%%%%%%%%%%%%%%%%%%
\section{Sequential Combination aus InverseCapacity und DemandFirstWaypoints}
% %%%%%%%%%%%%%%%%%%%%%%%%%%%%%%%%%%%%%%%%%%%%%%%%
% erste Folie
% %%%%%%%%%%%%%%%%%%%%%%%%%%%%%%%%%%%%%%%%%%%%%%%%
% zweite Folie
% %%%%%%%%%%%%%%%%%%%%%%%%%%%%%%%%%%%%%%%%%%%%%%%%
% dritte Folie
\begin{frame}{Zweiter Schritt: Motivation}
\Large
\begin{itemize}
    \item wenig Rechenzeit f\"ur $inverse\_capacity$ und $demand\_first\_waypoints$
    \item Kombi aus beidem genauer?
    \item zus\"atzliche Rechenzeit gerechtfertigt?
\end{itemize}
\end{frame}
% %%%%%%%%%%%%%%%%%%%%%%%%%%%%%%%%%%%%%%%%%%%%%%%%
% vierte Folie
\begin{frame}{Zweiter Schritt: Ergebnisse}
\Large
Zielmetrik: MLU (minimum link utilization)
\begin{center}
\includegraphics[width=.8\textwidth]{images/pouria1.png}
\end{center}
\end{frame}
\begin{frame}{Zweiter Schritt: Ergebnisse}
\Large
Zielmetrik: MLU (minimum link utilization) \textcolor{red}{+ Rechenzeit}
\begin{center}
\includegraphics[width=.8\textwidth]{images/pouria2.png}
\end{center}
\end{frame}
% %%%%%%%%%%%%%%%%%%%%%%%%%%%%%%%%%%%%%%%%%%%%%%%%
% neue Folien
% %%%%%%%%%%%%%%%%%%%%%%%%%%%%%%%%%%%%%%%%%%%%%%%%
\begin{frame}{Ergebnisse}
\begin{center}
\includegraphics[width=.8\textwidth]{images/pouria_all_algorithms_abilene.pdf}
\end{center}
\end{frame}
\begin{frame}{Ergebnisse}
\begin{center}
\includegraphics[width=.8\textwidth]{images/pouria_colored_scatter_plot_results_all_algorithms.pdf}
\end{center}
\end{frame}
\begin{frame}{Ergebnisse}
\begin{center}
\includegraphics[width=.8\textwidth]{images/pouria_real_demands.pdf}
\end{center}
\end{frame}
\begin{frame}{Ergebnisse}
\begin{center}
\includegraphics[width=.8\textwidth]{images/pouria_colored_scatter_plot_results_real_demands.pdf}
\end{center}
\end{frame}
\begin{frame}{Ergebnisse}
\begin{center}
\includegraphics[width=.8\textwidth]{images/pouria_colored_scatter_plot_results_all_topologies.pdf}
\end{center}
\end{frame}
% %%%%%%%%%%%%%%%%%%%%%%%%%%%%%%%%%%%%%%%%%%%%%%%%
% neue Folien
% %%%%%%%%%%%%%%%%%%%%%%%%%%%%%%%%%%%%%%%%%%%%%%%%
\begin{frame}{Zweiter Schritt: Fragen}
\Large
\begin{itemize}
    \item warum ist $SQ_{CW}$ besser als $SQ_{WC}$?
    \item Rechenaufwand von beiden gleich?
    \item warum $SQ$ so viel besser als $SQ_{CW}$ oder $SQ_{WC}$?
\end{itemize}
N\"achste Schritte:
\begin{itemize}
    \item Algos genauer analysieren \textcolor{lightgray}{Schritt 3}
    \item Experimente reproduzierbar machen \textcolor{lightgray}{Schritt 4}
\end{itemize}
\end{frame}
\begin{frame}[t,standout]
\Large
Fragen oder Anmerkungen?
\end{frame}

% %%%%%%%%%%%%%%%%%%%%%%%%%%%%%%%%%%%%%%%%%%%%%%%%%%%
% %%%%%%%%%%%%%%%%%%%%%%%%%%%%%%%%%%%%%%%%%%%%%%%%%%%
% Reproduktion
% %%%%%%%%%%%%%%%%%%%%%%%%%%%%%%%%%%%%%%%%%%%%%%%%%%%
% %%%%%%%%%%%%%%%%%%%%%%%%%%%%%%%%%%%%%%%%%%%%%%%%%%%
\section{Ergebnisse Reproduktion von Gruppe 3}
\begin{frame}{Frame Title}
\begin{center}
    \includegraphics[width=\textwidth]{images/err1.png}
\end{center}
\end{frame}
\begin{frame}{Frame Title}
\begin{center}
    \includegraphics[width=\textwidth]{images/err2.png}
\end{center}
\end{frame}
\begin{frame}{Frame Title}
\begin{center}
    \includegraphics[width=\textwidth]{images/err3.png}
\end{center}
\end{frame}
\begin{frame}{Frame Title}
\begin{center}
    \includegraphics[width=\textwidth]{images/err4.png}
\end{center}
\end{frame}
\begin{frame}{Frame Title}
\begin{center}
    \includegraphics[width=\textwidth]{images/err5.png}
\end{center}
\end{frame}
\end{document}
